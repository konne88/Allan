%%%%%%%%%%%%%%%%%%%%%%%%%%%%%%%%%%%%%%%%%%%%%%%%%%%%%%%%%%%%%%%%%%%%%%%%%%%%%%%%
% Description:  This file contains some useful package imports and document-   %
%               wide settings.                                                 %
%               The first line in the package comments always comprises the    %
%               official package caption as listed in the CTAN directory.      %
%               Additional lines explain why exactly I include this package.   %
%               Font choice and styling settings are to be regarded as highly  %
%               subjective and may be adjusted to personal taste.              %
%               Below settings work perfectly fine on my system (Windows XP +  %
%               TeXlipse + MikTeX 2.9). Only the droidsans package had to be   %
%               reinstalled manually because it was screwed up by MikTeX. But  %
%               you probably don't even like this font and will replace it     %
%               anyway.                                                        %
%               Use at your own risk. No warranties granted. Ignore subjective %
%               comments. Hope it helped ;-).                                  %
% Author:       Tobias Stöckel                                                 %
% Date:         08/02/2011                                                     %
%%%%%%%%%%%%%%%%%%%%%%%%%%%%%%%%%%%%%%%%%%%%%%%%%%%%%%%%%%%%%%%%%%%%%%%%%%%%%%%%


%%%%%%%%%%%%%%%%%%%%%%%%%%%%%%%%%%%%%%%%%%%%%%%%%%%%%%%%%%%%%%%%%%%%%%%%%%%%%%%%
% Packages

  \usepackage[utf8]{inputenc}
    % Accept different input encodings.
    % Needed to accept UTF-8 encoded source file.
  
  \usepackage[T1]{fontenc}
    % Standard package for selecting font encodings.
    % Needed for the droidsans font package.
  
  %\usepackage[defaultsans]{droidsans}
    % LaTeX support for the Droid Sans font family.
    % Make sure to install the droid package properly. MikTeX screws up the
    % font map by default. If you just want any sans-serif font, use the Latin
    % Modern Sans font. That one works out of the box (but looks ugly in the
    % bold version so setting the font for headings to regular may be a good
    % idea).
    
  \usepackage{lmodern}
    % The Latin Modern Family of Fonts
    % Improved version of Donald Knuth's original Computer Modern font. Shapes
    % are pretty much the same but glyph coverage has been extended.
  
  \usepackage{sectsty}
    % Control sectional headers.
    % Needed to have non-bold chapter and section headings.
  
  \usepackage{fullpage}
    % Set all page margins to 1.5cm.
    % Needed to comply with DHBW rules.
  
  \usepackage{longtable}
    % Allow tables to flow over page boundaries.
  
  \usepackage{parskip}
    % Layout with zero \parindent and non-zero \parskip.
    % Needed to turn off paragraph indentation and to customize paragraph
    % spacing.
  
  \usepackage{graphicx}
    % Enhanced support for graphics.
    % Replaces the old graphics package.
  
  \usepackage{float}
    % Improved interface for floating objects.
    % Needed for centering images on page (?).
  
  \usepackage{epstopdf}
    % Convert EPS to 'encapsulated' PDF using GhostScript.
    % Needed to embed EPS vector graphics. Make sure that GhostScript is
    % installed on your system (?).
  
  \usepackage{listings}
    % Typeset source code listings using LaTex.
    % Configure using \lstset.
    
  \usepackage{mdwlist}
    % Make lists more compact
    % Provides ``starred'' versions of the previous environments, e.g. itemize*,
    % which are more compact
    
  \usepackage{natbib}
    % Allows to choose between different citation styles, such as 'Harvard'
    % Requires bibliographystyle{plainnat}
    % Configure using \bibpunct
  
  \usepackage{tabularx}
    % Tabulars with adjustable-width columns.
    % Needed to easily stretch a table to a desired width (eg. page width)
  
  \usepackage[]{hyperref}
    % Extensive support for hypertext in LaTeX.
    % Needed to control the appearance of links in pdf and html output.
    % Configure using \hypersetup. Usually this package should appear at the end
    % of all other packages.
  
  \usepackage[all]{hypcap}
    % Adjusting the anchors of captions.
    % Prevents a link to an image from just jumping to the image's caption
    % instead of the image itself. Has to appear after the hyperref package.
  
  \usepackage[toc,acronym]{glossaries}
    % Create glossaries and lists of acronyms.
    % Has to be placed after the hyperref package to make links clickable


%%%%%%%%%%%%%%%%%%%%%%%%%%%%%%%%%%%%%%%%%%%%%%%%%%%%%%%%%%%%%%%%%%%%%%%%%%%%%%%%
% Settings

  % Configure the hyperref package.
    \hypersetup{
        unicode=false,                  % non-Latin characters in Acrobat’s bookmarks
        pdftoolbar=true,                % show Acrobat’s toolbar?
        pdfmenubar=true,                % show Acrobat’s menu?
        pdffitwindow=false,             % window fit to page when opened
        pdfstartview={FitH},            % fits the width of the page to the window
        pdftitle={Programmatic verification of a cooking recipe}, % title
        pdfauthor={Konstantin Weitz, Tobias Stöckel},     % author
        pdfsubject={Programmatic verification of a cooking recipe using allen-logic and LISP},
        pdfcreator={Tobias Stöckel},    % creator of the document
        pdfproducer={Tobias Stöckel},   % producer of the document
        pdfnewwindow=true,              % links in new window
        colorlinks=true,                % false: boxed links; true: colored links
        linkcolor=black,                % color of internal links
        citecolor=black,                % color of links to bibliography
        filecolor=black,                % color of file links
        urlcolor=black,                 % color of external links
        hypertexnames=false             % fix for ``duplicate page ref'' problem
    }

  % Configure the (source code) listings package.
    \lstset{ %
        language=SQL,                   % choose the language of the code
        basicstyle=\footnotesize,       % the size of the fonts that are used for the code
        numbers=left,                   % where to put the line-numbers
        numberstyle=\footnotesize,      % the size of the fonts that are used for the line-numbers
        stepnumber=2,                   % the step between two line-numbers. If it's 1 each line will be numbered
        numbersep=5pt,                  % how far the line-numbers are from the code
        backgroundcolor=\color{white},  % choose the background color. You must add \usepackage{color}
        showspaces=false,               % show spaces adding particular underscores
        showstringspaces=false,         % underline spaces within strings
        showtabs=false,                 % show tabs within strings adding particular underscores
        frame=single,                   % adds a frame around the code
        tabsize=2,                      % sets default tabsize to 2 spaces
        captionpos=b,                   % sets the caption-position to bottom
        breaklines=true,                % sets automatic line breaking
        breakatwhitespace=false,        % sets if automatic breaks should only happen at whitespace
        title=\lstname,                 % show the filename of files included with \lstinputlisting; also try caption instead of title
    }
  
  % Default search path for images (graphicx)
    \graphicspath{{images/}}
  
  % Set line spacing in order to comply with DHBW rules.
    \linespread{1.3}
    
  % Add paragraph spacing because it looks better - at least to me. And it
  % generates additional pages :-).
    \setlength{\parskip}{3ex plus 0.5ex minus 0.2ex}
  
  % Set the default font for this document to the sans-serif variant of the
  % currently loaded font package (?)
    \renewcommand*\familydefault{\sfdefault}
  
  % Set chapter and section headings to non-bold font. Again, just because it
  % looks better to me.
    \allsectionsfont{\normalfont}
  
  % Adjust interword spacing. The Droid Sans font needs some more of it in my
  % opinion.
  %  \spaceskip0.4em plus 1.66666pt minus 1.11111pt \relax
  
  % Adjust the natbib citation style, mainly forcing it to use parantheses
  % instead o f brackets.
    \bibpunct{(}{)}{;}{a}{,}{,}
  
  
