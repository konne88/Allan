\chapter{Preliminary Considerations}

    The theoretical basis for the following considerations is formulated
    by Allen's temporal logic. The individual steps of a recipe are activities
    with a specific length and with very specific temporal dependencies between each other. This
    makes them perfect candidates for a represenation according to Allen. A
    recipe in Allen-representation is a graph with one node per activity and with edges
    labelled by the temporal relations between them.
    
    In order to validate a recipe, its Allen-representation has to be checked
    for consistency. Luckily, only a few cases are important for this purpose:
    
    First of all, a graph consisting exclusively of isolated nodes
    is always consistent. This is because ``no edge'' between two nodes is equivalent to
    an edge labelled with all possible Allen-relations. Of all possible
    Allen-relations, at least one will always be satisfied, thus making it
    impossible to invalidate the edge.
    
    Secondly, a graph without any cycles is always consistent as well. If there
    are no cross-dependencies within the graph, no conflicts can arise between
    different edges. A cycle in that sense does not have to be directed.
    Although an Allen-graph has directed edges, they can be inverted at will by
    inverting their associated temporal relations.
    
    This leaves cycles as the critical component for consistency checking.
    Allen's logic provides an algorithm to validate the consistency of such
    cycles efficiently. The presented solution makes use of this
    algorithm. Unfortunately, it only works for cycles consisting of exactly
    three nodes. Because of that, this implementation scans a recipe for cycles
    first, and rejects it, if any cycle with more than three nodes is detected.
    
    Beyond verifying the consistency of a recipe, the solution also validates a
    list of concrete start and end times for the execution steps against the
    recipe. The basic notion is to infer the Allen-relations between the
    concrete execution steps and compare them to the Allen-relations in the
    recipe graph. If the relations for the inferred edges are subsets of the
    relations for the corresponding edges in the recipe graph, the concrete
    execution plan is a valid instantiation of the recipe. If any of the
    inferred relations is no subset of the corresponding relations in the
    recipe, the execution plan is invalid. This subset-checking is done by
    intersecting the execution plan graph with the recipe graph.
    
    For this
    intersection logic to work, it is important to note that a non-existent edge
    is fundamentally different from an edge with an empty set of relations.
    While the first one actually stands for ``all relations possible'', the
    second one stands for ``no relation possible''. An edge with an empty set of
    relations can never be satisfied, which invalidates the entire graph.

    As an aside, all input files are assumed to be syntactically and
    semantically correct (with the exception of inconsistencies in the temporal
    relations, of course). Especially for the execution plan file (times.csv),
    this means that each step has to be listed with exactly one start and one end time.