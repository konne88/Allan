\chapter{Evaluation}

  \section{Weaknesses}
  
    Even after careful design and development, the solution still carries some
    open points:
    
    From an end user's perspective, the application's usage is rather rough. Not
    only will he need to ensure a working Lisp environment, there is also no
    graphical user interface and the command line shows some potentially
    distracting internal noise. The handling of the input files might also be
    improved, considering flexible file names or syntax error handling.
    
    From a developer's perspective, the code organization could be optimized and
    unified, also considering formatting and naming conventions.
    Algorithmically, the cycle detection resorts to a brute-force approach,
    which might hold potential for performance optimizations.
  
  \section{Strengths}
  
    Despite its ``barrier to entry'', Lisp proves to be a very suitable tool for
    solving graph problems.
    
    The developed solution is believed to be a very powerful one, which can cope
    with arbitrarily complex input graphs (as long as computing capacity
    suffices and there are no cycles larger than three nodes). It works nicely
    with several cycles, separated graphs and isolated points.
    
    The source code itself is also sensibly modularized which improves
    understandiblity, testability und reusability.
 