\chapter{Implemenation}
\label{sec:impl}

    The solution is implemented entirely in Lisp. Lisp was chosen because of its
    strenghts in graph processing, as graphs can be easily represented as lists.
    The source code is split up into several files to allow for better
    modularization and collaboration. The main code takes care of importing all
    necessary files into \texttt{verfication.lsp}.
    
    \section{Workflow}
    
        The main processing workflow consists of two phases. At first, the
        recipe is checked for consistency:
        
        \begin{description}
            \item[Read recipe] \hfill \\
                Read the \texttt{relations.csv} input file, parse it and
                represent its tabular data as a nested list.
            \item[Create graph representation] \hfill \\
                Transform the tabular step data into a list of edges which are
                attributed with temporal relations.
            \item[Scan for cycles] \hfill \\
                Scan the graph for cycles by performing a depth-first-search.
                Only collect real cycles with no unnecessary additional nodes.
                The idea is, that in such a cycle, all nodes are connected and
                each node has exactly two different neighbors. Flag the graph as
                invalid if a cycle with four or more nodes is detected.
            \item[Normalize cycles] \hfill \\
                If necessary, invert some relations for the consistency check
                algorithm.
            \item[Check for local consistency] \hfill \\
                For each three-node cycle perform local consistency checking
                according to Allen's temporal logic. Makes use of the P-function
                as defined during classroom exercises.
        \end{description}
        
        Next, the execution plan is validated against the recipe:
        
        \begin{description}
            \item[Read execution plan] \hfill \\
                Read the \texttt{times.csv} input file, parse it and represent
                its tabular data as a nested list. Also perform some formatting,
                such as removing the colon from times to allow for arithmetic
                operations.
            \item[Infer temporal relations] \hfill \\
                Create a permutation of unordered pairs for the steps in the
                execution plan and infer the temporal relations for those pairs.
                There will be exactly one matching Allen relation per pair.
            \item[Create graph representation] \hfill \\
                The pairs from the previous step are again edges of a graph.
                Bring this graph to a common representation.
            \item[Intersect with recipe graph] \hfill \\
                For each edge, intersect the temporal relations from the recipe
                graph with those from the corresponding edge in the execution
                plan graph.
            \item[Check consistency of intersection] \hfill \\
                Use the previously defined graph consistency function to
                determine if the intersection of relations is still valid. If it
                is, then the execution plan is a valid instantiation of the
                recipe.
        \end{description}
    
    \section{Project Files}
    
        Quick reference of the project's files and directories:
        
        \begin{description}
            \item[.git] \hfill \\
                Version control
            \item[doc] \hfill \\
                Sources for this documentation
            \item[allan.lsp] \hfill \\
                Implementation of the basic allen-logic as developed during
                class-room exercises
            \item[graph.lsp] \hfill \\
                Graph-processing functions, mainly for cycle-detection
            \item[graph-tools.lsp] \hfill \\
                Functions for consistency checking and file-I/O
            \item[split-sequence.lsp] \hfill \\
                External library for processing the comma-separated files
            \item[verification.lsp] \hfill \\
                Functions for execution plan validation. Forms the application's
                entry point.
        \end{description}
    
    
    
    
    
    
    
    
    
    
    
    
    
    
    
    
    
    
    
    
    
    
    
    
    
    
    
    
    
    
    
    