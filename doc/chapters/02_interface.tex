\chapter{Public Interface}

  \section{Input Files}
    
    This implementation is accompanied by three input files. They are all
    plain-text files containing tabular data separated by semicolons. The first line of
    each file is ignored. In order to make the file more human-readable, this
    line contains a table header. The files and their usage are defined as
    follows:
    
    \begin{description}
        \item[relations.csv] \hfill \\
            Defines the recipe by listing the relations between the individual
            steps. The format of one line is:
            \texttt{<step\_1>;<step\_2>;<relation>}\\
            where steps are identified by numbers and the relation by the
            well-known symbols for Allen relations.
        \item[times.csv] \hfill \\
            Defines the concrete execution plan. For each step a
            \texttt{``Beginn''} and an \texttt{``Ende''} entry has to be
            present.
            The format of one line is:
            \texttt{<step>;[Beginn|Ende];<time>}\\
            Time is represented as a 24-hour value, such as \texttt{``18:45''}.
        \item[steps.csv] \hfill \\
            Defines the individual steps more closely. Assigns a description and
            a duration to the numbers used in the previous files. The format of
            one line is:\\
            \texttt{<step>;<description>;<duration>}\\
            The duration is given in minutes. This file is not used in this
            implementation because activity durations are derived from the times
            given in \texttt{times.csv}
    \end{description}
    
    The file names are fixed and the files have to be present in the code's
    working directory.
      
  \section{Usage}
    
    The entry point to the solution is the source file \texttt{verfication.lsp}
    (the ``i'' is actually missing from the file name ``for historical
    reasons'').
    It can be loaded to the Lisp environment either from a system command prompt
    via
    
    \begin{center}
        \texttt{clisp verfication.lsp}
    \end{center}
    
    or from the interactive interpreter or another Lisp program via
    
    \begin{center}\texttt{(load ``verfication.lsp'')}\end{center}
    
    Loading the file the one way or the other will automatically invoke the recipe
    and execution plan validation. Recipe validation can also be called manually
    through the function 
    
    \begin{center}\texttt{(check-relations (relations))}.\end{center}
    
    Execution plan
    validation can be called manually through the function
    
    \begin{center}\texttt{(check-execution-consistency (step-relations)
    (relations))}.\end{center}
    
    where \texttt{(relations)} reads the recipe's relations from
    \texttt{relations.csv} and \texttt{(step-relations)} reads the execution
    plan from \texttt{times.csv} and infers the relations between the
    activities.
  
  \section{Output}
    
    When loaded, the application will at first print some noise from internal
    test cases.
    In the last lines it will print verbally whether the recipe in
    \texttt{relations.csv} is consistent and whether the execution plan in
    \texttt{times.csv} matches the recipe.
    
    The two functions mentioned above return either \texttt{T} or \texttt{Nil},
    indicating either a successful or a failed validation.
