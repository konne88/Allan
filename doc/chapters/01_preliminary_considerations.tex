\chapter{Preliminary Considerations}

    The  following considerations are based on Allen's temporal logic. This
    approach is justified, because the individual steps of a recipe are
    activities with a specific length and with very specific temporal dependencies
    between them. A representation of a cooking recipe with multiple
    activities on a time axis can be found in figure \ref{fig:time-chart}. Such
    time intervals with temporal dependencies are ideal candidates for a
    represenation according to Allen. A recipe in Allen Representation can be
    thought of as a graph with one node
    per activity and with edges
    labelled with the temporal relations that hold between them. Those temporal
    relations mirror the dependencies between activities in a cooking recipe.
    The sample recipe from above is depicted in Allen Represenation in figure
    \ref{fig:graph}.
    
    \begin{figure}[]
     \centering
     \includegraphics[width=0.7\linewidth]{time-chart.pdf}
     \caption[Recipe time chart]{Representing a cooking recipe as a set of
     (possibly concurrent) activities.}
     \label{fig:time-chart}
    \end{figure}
    
    \begin{figure}[]
     \centering
     \includegraphics[width=0.7\linewidth]{recipe-graph.pdf}
     \caption[Recipe graph]{Representing the recipe from figure
     \ref{fig:time-chart} as an Allen Graph.}
     \label{fig:graph}
    \end{figure}
    
    \section{Recipe Validation}
    
    In order to validate a recipe, its Allen Representation has to be checked
    for consistency. Luckily, only a few cases have to be considered for this
    purpose:
    
    First of all, a graph consisting exclusively of \emph{isolated nodes}
    is guaranteed to be consistent. The reason for that is, that ``no edge''
    between two nodes is equivalent to an edge labelled with all possible
    Allen Relations. Of all possible Allen Relations, at least one will always
    be satisfied, thus making it impossible to invalidate the edge.
    
    Secondly, a graph \emph{without any cycles} is guaranteed to be consistent
    as well.
    If there are no cross-dependencies within the graph, no conflicts can arise between
    different edges. A cycle in that sense does not have to be directed.
    Although an Allen Graph has directed edges, they can be inverted at will by
    inverting their associated temporal relations.
    
    The two cases above leave \emph{cycles} as the critical component for
    consistency checking.
    Allen's logic provides an algorithm to validate the consistency of such
    cycles efficiently. This algorithm has been covered extensively during KBS
    courses, so it won't be elaborated on here. The presented solution makes use
    of this algorithm but unfortunately, it is only applicable to cycles
    consisting of exactly three nodes. As a consequence, this implementation
    %scans a recipe for cycles first, and rejects it, if any cycle with more
    %than three nodes is detected.
    only yields valid results for recipes that contain cycles with no more than
    three edges.
    
    \section{Execution Plan Validation}
    
    Beyond verifying the consistency of a recipe, the solution also validates a
    list of concrete start and end times for the individual activities against the
    recipe. This list will be referred to as the \emph{Execution Plan}.
    The basic idea behind this validation is to infer the Allen Relations between the
    concrete execution steps and compare them to the Allen Relations in the
    recipe graph. If the relations for the inferred edges are subsets of the
    relations for the corresponding edges in the recipe graph, the concrete
    execution plan is a valid instantiation of the recipe. If any of the
    inferred relations is no subset of the corresponding relations in the
    recipe, the execution plan is invalid. This subset-checking is done by
    intersecting the execution plan graph with the recipe graph.
    
    For this
    intersection logic to work, it is important to note that a non-existent edge
    is fundamentally different from an edge with an empty set of relations.
    While the first one actually stands for ``all relations possible'', the
    second one stands for ``no relation possible'', which are exact opposites.
    An edge with an empty set of
    relations can never be satisfied and thus invalidates the entire graph.

    As an aside, all input files are assumed to be syntactically and
    semantically correct (with the exception of inconsistencies in the temporal
    relations, of course). Especially for the execution plan file (\texttt{times.csv}),
    this means that each step has to be listed with exactly one start and one end time.